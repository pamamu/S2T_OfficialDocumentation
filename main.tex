\documentclass[12pt,a4paper,twoside]{book}

\usepackage[spanish, es-tabla]{babel}
\usepackage[utf8x]{inputenc}

\usepackage[table]{xcolor} % para poner color en las tablas

\usepackage{tikz} %Gráficos
\usetikzlibrary{calc}
\usetikzlibrary{arrows,shapes,positioning,shadows,trees}

\usepackage{amsmath} % formulas sin numero
\usepackage{float}

\usepackage{subcaption} %Varias imágenes, misma linea
\usepackage{graphicx} % para poner dos imágenes juntas
\graphicspath{{pictures/}{../pictures/}}


\usepackage{anysize}  %margenes
% MÁRGENES 
\marginsize{4cm}{2.5cm}{2.5cm}{2.5cm} 
%{izquierda}{derecha}{arriba}{abajo}

\usepackage{mathptmx}% http://ctan.org/pkg/mathptmx Times new roman
\usepackage{amsfonts}% Para expresiones matemáticas

\renewcommand{\baselinestretch}{1.5} % Interlineado
\usepackage{enumerate}% para los enumerate

\usepackage{fancyhdr} %header y footer
\pagestyle{fancy}

\usepackage[none]{hyphenat} %saltos y justificación texto

\usepackage{changepage} %margen y par/impar páginas

\usepackage{ragged2e} %Añade más funcionalidades de colocación de texto

\usepackage{listings} %Listas

\usepackage{appendix} %apéndice

\usepackage{color} %Colores

\makeatletter
\def\BState{\State\hskip-\ALG@thistlm}
\makeatother

%Ruta Imagenes

% COLORES
\definecolor{gray}{rgb}{0.4,0.4,0.4}
\definecolor{darkblue}{rgb}{0.0,0.0,0.6}
\definecolor{cyan}{rgb}{0.0,0.6,0.6}
\definecolor{palegreen}{rgb}{0.6, 0.98, 0.6}
\definecolor{palegoldenrod}{rgb}{0.93, 0.91, 0.67}

\colorlet{punct}{red!60!black}
\definecolor{background}{HTML}{EEEEEE}
\definecolor{delim}{RGB}{20,105,176}
\colorlet{numb}{magenta!60!black}


%Anexos Traducción
\renewcommand{\appendixname}{Anexos}
\renewcommand{\appendixtocname}{Anexos}
\renewcommand{\appendixpagename}{Anexos}

%Codigo fuente

\renewcommand{\lstlistingname}{Fragmento de código}

\lstset{
	basicstyle=\small\sffamily,
	numbers=left,
 	numberstyle=\tiny,
	frame=tb,
	tabsize=4,
	columns=fixed,
	showstringspaces=false,
	showtabs=false,
	keepspaces,
	commentstyle=\color{red},
	keywordstyle=\color{blue}
}

\lstdefinelanguage{json}{
    basicstyle=\footnotesize\ttfamily,
    numbers=left,
    numberstyle=\tiny,
    stepnumber=1,
    numbersep=8pt,
    showstringspaces=false,
    breaklines=true,
    frame=lines,
    backgroundcolor=\color{background},
    literate=
     *{0}{{{\color{numb}0}}}{1}
      {1}{{{\color{numb}1}}}{1}
      {2}{{{\color{numb}2}}}{1}
      {3}{{{\color{numb}3}}}{1}
      {4}{{{\color{numb}4}}}{1}
      {5}{{{\color{numb}5}}}{1}
      {6}{{{\color{numb}6}}}{1}
      {7}{{{\color{numb}7}}}{1}
      {8}{{{\color{numb}8}}}{1}
      {9}{{{\color{numb}9}}}{1}
      {:}{{{\color{punct}{:}}}}{1}
      {,}{{{\color{punct}{,}}}}{1}
      {\{}{{{\color{delim}{\{}}}}{1}
      {\}}{{{\color{delim}{\}}}}}{1}
      {[}{{{\color{delim}{[}}}}{1}
      {]}{{{\color{delim}{]}}}}{1},
}


\lstset{inputpath=code/, captionpos=b}
%Encabezado
\rhead{
    \begin{picture}(0,0) \put(0,0){\includegraphics[width=1cm]{logo_uex.png}}
    \end{picture}
}

%Variables
\title{Sistema de indexación de palabras clave e identificación de voz en señales de audio}
\author{Pablo Macías Muñoz}
\def\tutor{Roberto Rodríguez Echeverría} % co-author
\def\cotutor{Juan Carlos Preciado} % co-author %configuracion


\begin{document}
\sloppy 
\pagebreak 
%http://tex.stackexchange.com/questions/129088/create-a-frame-for-a-title-page
\makeatletter

\begin{titlepage}
	\begin{adjustwidth}{-1.5cm}{0cm}
		\begin{tikzpicture}[remember picture, overlay]
		\usetikzlibrary{calc}
		\draw[line width = 1.5pt] ($(current page.north west) + (1in,-1in)$) rectangle ($(current page.south east) + (-1in,1in)$);
		\end{tikzpicture}
		
		\vspace{-3.5em}
		\hspace{-0.5em}
		\begin{minipage}{0.45\textwidth}
			\begin{flushleft}
				\includegraphics[width=1.75cm]{logo_uex.png}
			\end{flushleft}
		\end{minipage}
		
		\vspace{-6em}
		\hspace{20em}
		\begin{minipage}{0.45\textwidth}
			\begin{flushright}
				\includegraphics[width=4cm]{logo_epcc.png}
			\end{flushright}
		\end{minipage}\\[1.5cm]
		
		\begin{center}
			
			
			% Upper part of the page
			\textsc{\LARGE UNIVERSIDAD DE EXTREMADURA}\\[4cm]
			
			\textsc{\Large Escuela Politécnica}\\[0.5cm]
			\textsc{\Large Grado de Ingeniería Informática en Ingeniería del Software}\\[3cm]
			\textsc{\Large Trabajo Fin de Grado}\\[0.5cm]
			{ \large \bfseries \@title}\\[4.0cm]
			\vfill
			% Bottom of the page
			{\large}
		\end{center}
	\end{adjustwidth}
\end{titlepage}

\pagestyle{empty}

\begin{titlepage}
	\begin{adjustwidth}{-1.5cm}{0cm}
		\begin{tikzpicture}[remember picture, overlay]
		\usetikzlibrary{calc}
		\draw[line width = 1.5pt] ($(current page.north west) + (1in,-1in)$) rectangle ($(current page.south east) + (-1in,1in)$);
		\end{tikzpicture}
		
		\vspace{-3.5em}
		\hspace{-0.5em}
		\begin{minipage}{0.45\textwidth}
			\begin{flushleft}
				\includegraphics[width=1.75cm]{logo_uex.png}
			\end{flushleft}
		\end{minipage}
		
		\vspace{-6em}
		\hspace{20em}
		\begin{minipage}{0.45\textwidth}
			\begin{flushright}
				\includegraphics[width=4cm]{logo_epcc.png}
			\end{flushright}
		\end{minipage}\\[1.5cm]
		
		\begin{center}
			
			
			% Upper part of the page
			\textsc{\LARGE UNIVERSIDAD DE EXTREMADURA}\\[3cm]
			
			\textsc{\Large Escuela Politécnica}\\[0.5cm]
			\textsc{\Large Grado de Ingeniería Informática en Ingeniería del Software}\\[2.5cm]
			\textsc{\Large Trabajo Fin de Grado}\\[0.5cm]
			{ \large \bfseries \@title}\\[4.0cm]			
			{ \large \bfseries Autor: \@author}\\[0.5cm]
			{ \large \bfseries Tutor: \tutor}\\[0.5cm]
			{ \large \bfseries Co-Tutor: \cotutor}\\[0.5cm]
			\vfill
			% Bottom of the page
			{\large}
		\end{center}
	\end{adjustwidth}
\end{titlepage}

\makeatother
\thispagestyle{empty}
\newpage
\thispagestyle{empty}

\documentclass[../main.tex]{subfiles}
\begin{document}

\chapter*{Resumen}\label{ch:es_abstract}
\thispagestyle{empty}
Los sistemas de reconocimiento de voz se encuentran en pleno crecimiento debido a la aparición de nuevas herramientas de inteligencia artificial y aprendizaje automático. Los sistemas más usados se encuentran online y son propiedad de las empresas más grandes a nivel mundial del sector TIC. Además ninguna de estas herramientas permite la adaptabilidad y mejora de modelos, despliegue personalizado y usabilidad por cualquier usuario. De esta necesidad nace este proyecto. 

Este documento describe el proceso seguido en el diseño e implementación de un sistema capaz de reconocer la voz en un audio, además de ser mejorable introduciendo más información, adaptable a cualquier entorno acústico, desplegable en cualquier servidor y usable por la mayoría de usuarios. 

Al finalizarse la lectura de este documento no solamente se entenderá la estructura del proyecto, sino que se conocerá la historia de los sistemas de reconocimiento de voz, cuáles son y en qué se basan las actuales herramientas que existen, qué objetivos cumple el sistema propuesto, qué tecnologías utiliza, cómo se ha desarrollado el proyecto y el resultado de algunas pruebas realizadas.

\smallskip
\noindent \textbf{Palabras clave.} lingüística computacional, voz a texto, entrenamiento de modelos, microservicios


\end{document}

\documentclass[../main.tex]{subfiles}
\begin{document}

\chapter*{Abstract}\label{ch:en_abstract}
\thispagestyle{empty}
EL RESUMEN EN INGLÉS

\end{document}



\pagenumbering{roman}
\setcounter{page}{1}
\tableofcontents
%\pagebreak
\newpage
\listoftables
\newpage
\listoffigures
\newpage


%%%%%%% ESTRUCTURA DEL TFG %%%%%%%%%%%%%%
% A.PORTADA (según la estructura indicada a continuación) 
% B.CONTRAPORTADA (según la estructura indicada a continuación) 
% C.ÍNDICE GENERAL DE CONTENIDOS  
% D.ÍNDICE DE TABLAS  
% E.ÍNDICE DE FIGURAS 
% F.RESUMEN (Podrá incluirse también en inglés, si así lo indica el Tutor1) 
% G.CUERPO DEL TRABAJO (según la estr
% uctura indicada a continuación) 
% H.REFERENCIAS BIBLIOGRÁFICAS 
% (Según norma ISO690)
% I.ANEXOS, si los hubiera
%%%%%%%%%%%%%%%%%%%%%%%%%%%%%%%%%%%%%%%
%%%%%% Cuerpo del trabajo
% 1.INTRODUCCIÓN 
% 2.OBJETIVOS 
% 3.ANTECEDENTES / ESTADO DEL ARTE 
% 4.MÉTODOLOGÍA 
% 5.IMPLEMENTACIÓN Y DESARROLLO (Cuando proceda) 
% 6.RESULTADOS Y DISCUSIÓN 
% 7.CONCLUSIONES 
%%%%%%%%%%%%%%%%%%%%%%%%%%%%%%%%%%%%%%%%%%%
%%%%%%%%%%%%%%%%%%%%%%%%%%%%%%%%%%%%%%%%%%%
% Tamaño: Normalizado UNE A-4, salvo planos. 
% Tipo y tamaño de letra del texto: Times New Roman 12 pt, Arial 12 pt o similar. 
% Interlineado del texto: 1,5 líneas. 
% Márgenes del texto: Superior, Inferior y Derecha, 2,5 cm; Izquierda, 4 cm. 
% Numeración de páginas en margen inferior derecha y tamaño 8 pt. 
% Las  figuras  serán  numeradas  y  tituladas  debajo  de  las  mismas  (indicando  su  fuente  si  no  son  de  elaboración  
% propia). 
% Las  tablas  serán  numeradas  y  tituladas  encima  de  las  mismas  (indicando  su  fuente  si  no  son  de  elaboración  
% propia). 
%%%%%%%%%%%%%%%%%%%%%%%%%%%%%%%%%%%%%%%%%%%
%%%%%%%%%%%%%%%%%%%%%%%%%%%%%%%%%%%%%%%%%%%
%%%%%%%%%%%%%%%%%%%%%%%%%%%%%%%%%%%%%%%%%%%
%%%%%%%%%%%%%%%%%%%%%%%%%%%%%%%%%%%%%%%%%%%


\pagenumbering{arabic}
\pagestyle{fancy}
\setcounter{page}{1}

\chapter{Introducción}
Esta plantilla sirve como ejemplo de TFG. Las secciones son las que están en la normativa. Esta sección la aprovechamos para introducir brevemente \LaTeX.
\par

\section{Secciones}
Las secciones se crean con \textit{section}. Las subsecciones y subsubsecciones con \textit{subsection} y \textit{subsubsection}, respectivamente. Si se desea que alguna subsección en concreto no salga en el índice se pueden usar los mismos comandos añadiéndoles un asterisco al final (\textit{subsection*} y \textit{subsubsection*}).
\par
Lorem ipsum dolor sit amet, consectetur adipiscing elit, sed eiusmod tempor incidunt ut labore et dolore magna aliqua. Ut enim ad minim veniam, quis nostrud exercitation ullamco laboris nisi ut aliquid ex ea commodi consequat. Quis aute iure reprehenderit in voluptate velit esse cillum dolore eu fugiat nulla pariatur. Excepteur sint obcaecat cupiditat non proident, sunt in culpa qui officia deserunt mollit anim id est laborum.
\subsection{una subsección}
Ut enim ad minim veniam, quis nostrud exercitation ullamco laboris nisi ut aliquid ex ea commodi consequat. Quis aute iure reprehenderit in voluptate velit esse cillum dolore eu fugiat nulla pariatur.
\subsection{otra subsección}
Excepteur sint obcaecat cupiditat non proident, sunt in culpa qui officia deserunt mollit anim id est laborum.
\subsubsection{subsubsecciones}
¡Por defecto las subsubsecciones no aparecen en el índice! Este comportamiento se puede cambiar.

\section{Citas y referencias}
Para citar un texto hay que incluirlo en el fichero \textit{library.bib} de esta plantilla. Una vez hecho se puede\cite{CMUSphinx} referenciar usando el comando \textit{cite}~\cite{LaTeX_tutorials}.
Para referenciar imágenes, secciones o tablas se usa el comando~\textit{ref}. Por ejemplo~\ref{fig:logoEpcc}. Es importante haber añadido una etiqueta con el nombre correspondiente al emenento a referenciar.
\par
Para facilitar la lectura, cuando se insertan citas y referencias es conveniente insertar un caracter de espacio sin salto de línea \textit{\~} antes del comando de cita o referencia.


\section{Imágenes}
Las imágenes se insertan así:

% \begin{figure}[H]
% \centering
% \includegraphics[width=0.6\textwidth]{logo_epcc.png}}
% \caption[Logo Epcc]{Logo Epcc.\\Fuente:http://www.unex.es/conoce-la-uex/centros/epcc/}
% \label{fig:logoEpcc}
% \end{figure}


\chapter{Objetivos}
Lorem ipsum dolor sit amet, consectetur adipiscing elit, sed eiusmod tempor incidunt ut labore et dolore magna aliqua. Ut enim ad minim veniam, quis nostrud exercitation ullamco laboris nisi ut aliquid ex ea commodi consequat. Quis aute iure reprehenderit in voluptate velit esse cillum dolore eu fugiat nulla pariatur. Excepteur sint obcaecat cupiditat non proident, sunt in culpa qui officia deserunt mollit anim id est laborum.

\chapter{Estado del Arte}
Lorem ipsum dolor sit amet, consectetur adipiscing elit, sed eiusmod tempor incidunt ut labore et dolore magna aliqua. Ut enim ad minim veniam, quis nostrud exercitation ullamco laboris nisi ut aliquid ex ea commodi consequat. Quis aute iure reprehenderit in voluptate velit esse cillum dolore eu fugiat nulla pariatur. Excepteur sint obcaecat cupiditat non proident, sunt in culpa qui officia deserunt mollit anim id est laborum.

\chapter{Metodología}
Lorem ipsum dolor sit amet, consectetur adipiscing elit, sed eiusmod tempor incidunt ut labore et dolore magna aliqua. Ut enim ad minim veniam, quis nostrud exercitation ullamco laboris nisi ut aliquid ex ea commodi consequat. Quis aute iure reprehenderit in voluptate velit esse cillum dolore eu fugiat nulla pariatur. Excepteur sint obcaecat cupiditat non proident, sunt in culpa qui officia deserunt mollit anim id est laborum.

\chapter{Implementación y desarrollo}
Lorem ipsum dolor sit amet, consectetur adipiscing elit, sed eiusmod tempor incidunt ut labore et dolore magna aliqua. Ut enim ad minim veniam, quis nostrud exercitation ullamco laboris nisi ut aliquid ex ea commodi consequat. Quis aute iure reprehenderit in voluptate velit esse cillum dolore eu fugiat nulla pariatur. Excepteur sint obcaecat cupiditat non proident, sunt in culpa qui officia deserunt mollit anim id est laborum.

\chapter{Resultados}
Lorem ipsum dolor sit amet, consectetur adipiscing elit, sed eiusmod tempor incidunt ut labore et dolore magna aliqua. Ut enim ad minim veniam, quis nostrud exercitation ullamco laboris nisi ut aliquid ex ea commodi consequat. Quis aute iure reprehenderit in voluptate velit esse cillum dolore eu fugiat nulla pariatur. Excepteur sint obcaecat cupiditat non proident, sunt in culpa qui officia deserunt mollit anim id est laborum.

\chapter{Conclusiones y trabajos futuros}
Lorem ipsum dolor sit amet, consectetur adipiscing elit, sed eiusmod tempor incidunt ut labore et dolore magna aliqua. Ut enim ad minim veniam, quis nostrud exercitation ullamco laboris nisi ut aliquid ex ea commodi consequat. Quis aute iure reprehenderit in voluptate velit esse cillum dolore eu fugiat nulla pariatur. Excepteur sint obcaecat cupiditat non proident, sunt in culpa qui officia deserunt mollit anim id est laborum.


%%%%%%%%%%%%%%%%%%%%%%%%%%%%%%%%%%
%%%%%%%%%%% ANEXOS %%%%%%%%%%%%%%%
%%%%%%%%%%%%%%%%%%%%%%%%%%%%%%%%%%
\appendix
\clearpage
\appendixpage
\addappheadtotoc

\chapter{Ejemplo de anexo}
Si no se desea incluir anexos, sólo hay que borrar este capítulo.
\par
Lorem ipsum dolor sit amet, consectetur adipiscing elit, sed eiusmod tempor
incidunt ut labore et dolore magna aliqua. Ut enim ad minim veniam, quis nostrud exercitation ullamco laboris nisi ut aliquid ex ea commodi consequat. Quis aute iure reprehenderit in voluptate velit esse cillum dolore eu fugiat nulla pariatur. Excepteur sint obcaecat cupiditat non proident, sunt in culpa qui officia deserunt mollit anim id est laborum.\\


\pagebreak
%%%%%%%%%%%%%%%%%%%%%%%%%%%%%%%%%%
%%%%%%%%%% AL FINAL %%%%%%%%%%%%%%
%%%%%%%%%%%%%%%%%%%%%%%%%%%%%%%%%%
\thispagestyle{empty}
\pagestyle{empty}
%%%% https://en.wikibooks.org/wiki/LaTeX/Bibliography_Management
\addcontentsline{toc}{chapter}{Bibliografía}
\bibliographystyle{unsrt}
\bibliography{library.bib}
\end{document}
