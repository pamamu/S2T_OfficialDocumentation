\documentclass[../main.tex]{subfiles}
\begin{document}

\chapter{Propuesta de solución}\label{ch:propuesta_solucion}
En este apartado el autor hace una descripción de la solución propuesta para alcanzar los objetivos definidos en el \autoref{ch:objetivos}.

%%%%%%%%%%%%%%%%%%%%%%%%%%%%%%%%%%
%%      Análisis y Diseño       %%
%%%%%%%%%%%%%%%%%%%%%%%%%%%%%%%%%%
\section{Análisis y diseño}\label{sec:analisis_diseno}
En primer lugar se describe brevemente los requisitos del sistema junto a un análisis de los casos de uso. Posteriormente se describe el diseño arquitectónico del sistema completo y cómo se comunican los distintos componentes que lo forman.

%%%%%%%%%%%%%%%%%%%%%%%%%%%%%%%%%%
%%   Análisis de Requisitos     %%
%%%%%%%%%%%%%%%%%%%%%%%%%%%%%%%%%%
\subsection{Análisis de Requisitos}\label{subsec:analisis_requisitos}
A continuación se exponen los requisitos funcionales, no funcionales, de información y restricciones del sistema.

%%%%%%%%%%%%%%%%%%%%%%%%%%%%%%%%%%
%%   Requisitos Funcionales     %%
%%%%%%%%%%%%%%%%%%%%%%%%%%%%%%%%%%
\subsubsection{Requisitos funcionales}\label{subsubsec:req_funcionales}
Los requisitos funcionales del sistema son:
\begin{enumerate}
    \item 
\end{enumerate}

%%%%%%%%%%%%%%%%%%%%%%%%%%%%%%%%%%
%%  Requisitos no funcionales   %%
%%%%%%%%%%%%%%%%%%%%%%%%%%%%%%%%%%
\subsubsection{Requisitos no funcionales}\label{subsubsec:req_nofuncionales}
Los requisitos no funcionales del sistema son:
\begin{enumerate}
    \item 
\end{enumerate}

%%%%%%%%%%%%%%%%%%%%%%%%%%%%%%%%%%
%%  Requisitos de información   %%
%%%%%%%%%%%%%%%%%%%%%%%%%%%%%%%%%%
\subsubsection{Requisitos de información}\label{subsubsec:req_informacion}
Los requisitos de información del sistema son:
\begin{enumerate}
    \item 
\end{enumerate}

%%%%%%%%%%%%%%%%%%%%%%%%%%%%%%%%%%
%%        Restricciones         %%
%%%%%%%%%%%%%%%%%%%%%%%%%%%%%%%%%%
\subsubsection{Restricciones}\label{subsubsec:restricciones}
Las restricciones del sistema son:
\begin{enumerate}
    \item 
\end{enumerate}

%%%%%%%%%%%%%%%%%%%%%%%%%%%%%%%%%%
%%         Casos de Uso         %%
%%%%%%%%%%%%%%%%%%%%%%%%%%%%%%%%%%
\subsection{Casos de uso}\label{subsec:casos_uso}

%%%%%%%%%%%%%%%%%%%%%%%%%%%%%%%%%%
%%         Arquitectura         %%
%%%%%%%%%%%%%%%%%%%%%%%%%%%%%%%%%%
\subsection{Arquitectura del sistema}\label{subsec:arquitectura_sistema}
El sistema está formado por los siguientes componentes:
\paragraph{MainController}
\paragraph{GetAudioTrans}
\paragraph{AudioProcess}
\paragraph{Training}
\paragraph{G2P}
\paragraph{SRILM}
\paragraph{SPHINXBASE}
\paragraph{Speech2Text}

%%%%%%%%%%%%%%%%%%%%%%%%%%%%%%%%%%
%% Comunicación de componentes  %%
%%%%%%%%%%%%%%%%%%%%%%%%%%%%%%%%%%
\subsection{Comunicación de componentes}\label{subsec:comunicacion_componentes}

\paragraph{Diagrama de secuencia de obtención de fichero de audio y transcripción}
Prueba
\begin{figure}[htbp]
  \centering
  \includesvg[width=\linewidth, svgpath = ....pictures/]{containers_architecture}
  \caption{svg image}
\end{figure}
\begin{figure}[H]
  \includegraphics[width=\linewidth]{pictures/logo_epcc.png} %linewidth -> Ancho de la página
  \caption{A boat.} %Texto debajo de la imagen
  \label{fig:boat1} %Usado para referenciar
\end{figure}
\paragraph{Diagrama de secuencia de entrenamiento de modelos}
prueba
\paragraph{Diagrama de secuencia de proceso de transcripción}
prueba

%%%%%%%%%%%%%%%%%%%%%%%%%%%%%%%%%%
%%      Implementación          %%
%%%%%%%%%%%%%%%%%%%%%%%%%%%%%%%%%%
\section{Implementación y desarrollo}\label{sec:implementacion}

\end{document}