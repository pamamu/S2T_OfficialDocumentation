\documentclass[../main.tex]{subfiles}
\begin{document}

\chapter{Introducción}\label{ch:introduccion}
Da la sensación de que el reconocimiento de voz ha aparecido en estos últimos años. Asistentes por voz, escritura por voz en dispositivos, traducción automática por voz y más aplicaciones y herramientas que están por aparecer, son las señales que dan a entender que las grandes compañías como Google, Apple, IBM, Amazon Web Services o Microsoft tienen como línea de investigación y desarrollo el tratamiento y reconocimiento de la voz humana.

El reconocimiento de voz tuvo su aparición en el año 1952 cuando tres investigadores de Bell Labs construyeron un sistema llamado ``Audrey`` para el reconocimiento de dígitos por un solo hablante. Este sistema utilizaba el espectro de potencia que generaba el hablante con la pronunciación de cada dígito.

Durante varios años los investigadores fueron desarrollando nuevos sistemas de reconocimiento de voz hasta que en la década de 1980 salieron a la luz los modelos de lenguajes basados en probabilidades de aparición de palabras.

Ya en la última década del milenio pasado, Xuedong Huang desarrolló el sistema Sphinx-II en la Universidad de Carnegie Mellon (siglas CMU en inglés). Este sistema fue el primero en hacer reconocimiento de voz contínuo, independiente del hablante y con un gran vocabulario. Esto fue un hito bastante importante en la historia ya que nunca antes un sistema había manejado el habla contínua con un vocabulario extenso.

En los últimos años se ha avanzado exponencialmente en este campo dotando de inteligencia a estos sistemas con los avances en tecnologías de Machine Learning y Deep Learning. 

Después de tantos años de investigación y desarrollo en esta rama de la lingüística computacional, la mayoría de los sistemas de transcripción de voz a texto son privados y de pago ya que han sido comprados por grandes multinacionales como L\&H, ScanSoft o Apple.

El presente proyecto ofrecerá a los usuarios sin conocimientos técnicos una herramienta con la que utilizar servicios de reconocimiento de voz para realizar transcripciones, además de un sistema que permite entrenar los modelos que se utilizan en estos procesos para mejorar la calidad de los resultados. 
\end{document}