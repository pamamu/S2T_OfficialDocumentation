\documentclass[../main.tex]{subfiles}
\begin{document}

\chapter*{Abstract}\label{ch:en_abstract}
\thispagestyle{empty}
Speech recognition systems are in full growth due to the emergence of artificial intelligence and new automatic learning tools. The most widely used systems are online and are owned by the world's largest companies in the ICT sector. In addition, none of these tools allows adaptability and improvement of models, custom deployment and usability by any user. From this need is born this project. 

This document describes the process followed in the design and implementation of a system capable of recognizing the voice in an audio, in addition to be improved by introducing more information, adaptable to any acoustic environment, deployable on any server and usable by most users. 

When the reading of this document is finished, not only the structure of the project will be understood, but also the history of the voice recognition systems, what are the current tools that exist, on what are the existing tools based on, what objectives the proposed system fulfills, what technologies it uses, how the project has been developed and the result of some tests carried out.

\smallskip
\noindent \textbf{Keywords.} computational linguistics, speech to text, model trainin, microservices

\end{document}