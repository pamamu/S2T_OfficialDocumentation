\documentclass[../main.tex]{subfiles}
\begin{document}

\chapter*{Resumen}\label{ch:es_abstract}
\thispagestyle{empty}
Los sistemas de reconocimiento de voz se encuentran en pleno crecimiento debido a la aparición de nuevas herramientas de inteligencia artificial y aprendizaje automático. Los sistemas más usados se encuentran online y son propiedad de las empresas más grandes a nivel mundial del sector TIC. Además ninguna de estas herramientas permite la adaptabilidad y mejora de modelos, despliegue personalizado y usabilidad por cualquier usuario. De esta necesidad nace este proyecto. 

Este documento describe el proceso seguido en el diseño e implementación de un sistema capaz de reconocer la voz en un audio, además de ser mejorable introduciendo más información, adaptable a cualquier entorno acústico, desplegable en cualquier servidor y usable por la mayoría de usuarios. 

Al finalizarse la lectura de este documento no solamente se entenderá la estructura del proyecto, sino que se conocerá la historia de los sistemas de reconocimiento de voz, cuáles son y en qué se basan las actuales herramientas que existen, qué objetivos cumple el sistema propuesto, qué tecnologías utiliza, cómo se ha desarrollado el proyecto y el resultado de algunas pruebas realizadas.

\smallskip
\noindent \textbf{Palabras clave.} lingüística computacional, voz a texto, entrenamiento de modelos, microservicios


\end{document}